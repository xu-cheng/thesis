\chapter{Introduction}\label{chap:intro}

\section{Background}

With the recent advances in data-as-a-service (DaaS) and cloud computing, outsourcing data to the cloud has become a common practice. In a typical scenario shown in \Cref{fig:intro:model}, the \emph{data owner} (DO) outsources the data to a third party in the cloud known as \emph{service provider} (SP). Upon receiving the query requests from the clients, the SP will process them on the behalf of the DO\@. Since the SP has more computation power and storage capability than the DO, such a cloud outsourcing paradigm is able to scale the services with a low cost. However, as the SP is often a third-party delegate of the DO, it cannot be fully trusted. For example, the SP may return tampered or partial results due to a variety of reasons such as service outages, hack attacks, or even corporate dishonesty. Therefore, it is important to authenticate the query results such that the clients can verify their soundness (all results are genuine) and completeness (no result is missing).

\begin{figure}[h]
  \centering
  \resizebox{.7\linewidth}{!}{\begin{tikzpicture}
  \node [scale=1] (do) {\includegraphics{./figs/icons/database.eps}};
  \node [scale=1,right=2cm of do](sp){\includegraphics{./figs/icons/server.eps}};
  \node [scale=2,right=2cm of sp](client){\includegraphics{./figs/icons/user.eps}};
  \draw[-latex] (do.east) -- (sp.west) node [midway,above,font=\scriptsize] {Data \& ADS};
  \draw[-latex,transform canvas={yshift=0.5ex}] (client.west) -- (sp.east) node [midway,above,font=\scriptsize] {Queries};
  \draw[-latex,transform canvas={yshift=-0.5ex}] (sp.east) -- (client.west) node [midway,below,font=\scriptsize] {Results \& VO};
  \node [font=\footnotesize,below=0cm of sp](sp-label){\textbf{Services Provider (SP)}};
  \node [font=\footnotesize] at (do |- sp-label) {\textbf{Data Owner (DO)}};
  \node [font=\footnotesize] at (client |- sp-label) {\textbf{Clients}};
\end{tikzpicture}
}
  \caption{Authenticated Query Processing System Model}\label{fig:intro:model}
\end{figure}

In the field of query processing, query authentication has been studied by a large body of literature~\cite{10.1109/ICDE.2004.1320027,10.1145/1142473.1142488,10.1007/s00778-008-0113-2,10.1145/1880022.1880026,10.1145/2213836.2213871,10.1145/2463676.2465281,10.14778/2732219.2732224,10.1145/2664243.2664244,10.1145/2723372.2747649,10.1109/tkde.2014.2316818}. The basic idea is that the DO signs a well-designed \emph{authenticated data structure} (ADS) as notarization of the outsourced data. Based on the ADS, the SP then constructs a cryptographic proof, known as \emph{verification object} (VO), for each query, and returns it along with the query results. The VO can be used by the clients to verify the correctness of the query results.

However, the prior works have only considered limited query types. They fail to address a variety of needs demanded by enterprise customers. For example, aggregate queries over set-valued data used in data analytics play an important role in domains such as business decision making, scientific research, and government policy study. Yet, the majority of the existing query authentication algorithms focus on relational or spatial data and their associated query types. Further, none of the existing solutions considers the data confidentiality requirement, which is often needed in order to protect business secrets, preserve user privacy, and comply with regulations. Similarly, access control, another important security concern, is also largely ignored by the existing works. Finally, most of the existing studies on query authentication are confined to a centralized environment. Little research is conducted to investigate authenticated query processing in a distributed setting.

To address these issues, in this dissertation, we take the first step to study three new problems, namely
\begin{inlineenum}
  \item authentication of aggregate queries over set-valued data,
  \item authentication of relational queries with fine-grained access control, and
  \item authentication of {kNN} queries in distributed settings.
\end{inlineenum}

\section{Authenticating Aggregate Queries over Set-Valued Data}

Aggregate query services over set-valued data are becoming widely available for business intelligence, scientific research, and government policy study. For example, personal genomics analysis (e.g., 23andMe and the Personal Genome Project (PGP) at Harvard Medical School~\cite{pgp}) is based on aggregate queries on large genome datasets, the integrity of whose results is vital. Below is one example:

\begin{table}[t]
  \centering
  \begin{tabular}{ccl}
    \toprule
    \textbf{PID} & \textbf{ZIP} & \multicolumn{1}{c}{\textbf{Mut-Genes}}\\
    \midrule
    P1&95014&A-C130R, P-I696M\\
    P2&20482&H-C282Y, P-P12A, R-G1886S \\
    P3&95014&A-C130R, U-G71R, W-R611H\\
    P4&01720&A-V2050L, H-C282Y, M-R52C, U-G71R\\
    P5&20134&A-C130R, P-P12A, R-G1886S, S-E366K\\
    P6&17868& C-R102G, R-G1886S\\
    P7&55410&C-R102G, C-Q1334H, S-E288V\\
    P8&20852&C-R102G, P-P12A, R-G1886S, K-T220M\\
    \bottomrule
  \end{tabular}
  \caption{Set-Valued Genome Dataset}\label{tab:intro:sample_pgp}
\end{table}

\begin{example}[Aggregate Queries on PGP Data]~\label{example:intro:pgp}
  \Cref{tab:intro:sample_pgp} shows a sample genome dataset, where \emph{PID} is participant ID, \emph{ZIP} is ZIP code, and \emph{Mut-Genes} is a sensitive set-valued attribute that records the mutation genes of each participant. The query clients (e.g., medical doctors) may be interested in the following aggregate queries:
    \begin{itemize}
      \item \textbf{Q1}: Find the most common gene in the district of Cupertino, CA (ZIP\@: 95014).
      \item \textbf{Q2}: Count the number of participants who carry the gene `R-G1886S'.
      \item \textbf{Q3}: Find the most frequent genes with supports $\ge$ 3 in ZIPs 20***.
    \end{itemize}
    The corresponding query results are: \{`A-C130R'\}, 4, and \{`P-P12A', `R-G1886S'\}, respectively.
\end{example}

Although there are extensive studies on authenticated query processing, the prior research cannot be applied to aggregate queries on set-valued genomics or business data as exemplified above for two reasons. First, most of the research has been focused on relational or spatial data and their associated query types, whereas set-valued data have their own unique operations and aggregate query types. Second, most of the research ignores data confidentiality --- they assume that the SP is willing to and capable of sharing a substantial amount of source data to the query client for result verification. In the genomics analysis, business intelligence, and many other cases, however, the source data cannot be disclosed to the query client, either because they are private assets of the data owner, or because they may contain sensitive personal information such as individual genome sequences.

As set-valued data are ubiquitous and sometimes indispensable in describing real-world problems for analytics, in this dissertation we study authenticated aggregate query services over set-valued data with confidentiality preservation. We assume that a dataset consists of one sensitive set-valued attribute (e.g., mutation-gene set) and multiple non-sensitive attributes (e.g., ZIP code and age). As illustrated in \Cref{example:intro:pgp}, an aggregate query is defined as a query whose result is derived from aggregates of data. Particularly, each aggregate query consists of two phases: a filtering phase that filters data by a range selection on non-sensitive attributes, followed by an aggregating phase that aggregates on the sensitive set-valued attribute. For broader applicability, we model the sensitive set-valued attribute as a \emph{multiset}, i.e., a set that allows duplicate elements. This enables advanced data analytics such as frequent itemset mining in market basket analysis~\cite{Agrawal:1994:FAM:645920.672836}, inverted indexes in web search and auto-completion~\cite{10.1145/1148170.1148234}, and web log mining~\cite{10.14778/2212351.2212353}, all of which require a collection data type that supports duplicate elements.

Because aggregate queries exhibit unique properties that are unseen in those queries previously studied for authentication~\cite{10.1109/ICDE.2004.1320027,10.1145/1142473.1142488,10.1007/s00778-008-0113-2}, the design of the ADS and its related authentication techniques face great challenges. First, due to the dynamics of aggregate queries (e.g., each query in \Cref{example:intro:pgp} selects a different group of participants), it is impossible to pre-compute the query results and endorse them in advance. Second, aggregate queries over set-valued data are largely supported by multiset operations, for which existing authentication techniques, mostly designed for relational queries, cannot be applied. Third, to safeguard source data confidentiality, the query client must be able to verify the query results without learning any single set of sensitive values. To address these challenges, we propose PA$^2$, a privacy-preserving authentication framework for aggregate queries. It consists of an authenticated index \emph{Merkle Grid tree} (MG-tree), a family of privacy-preserving authentication protocols for primitive multiset operations, and a series of aggregate query processing algorithms. We further propose several optimizations to reduce the computation and communication overheads for both the server and the query client.

\section{Authenticating Relational Queries with Fine-Grained Access Control}

In addition to data integrity, access control, another important security concern, is also crucial to enterprise customers. Access control ensures each client can access only the data he/she is authorized to use. With the increasing popularity of moving enterprise databases into the cloud, the need for access control in data sharing is becoming more indispensable than ever. For example, to support cloud-based ERP and OLAP systems, a Salesforce cloud database can be configured to support up to $\fnum{10000}$ access control polices~\cite{salesforce}. To this end, \emph{attribute-based encryption} (ABE)~\cite{10.1145/1180405.1180418,10.1109/sp.2007.11} has been a prevailing technique used by cloud databases to support \emph{fine-grained} access control~\cite{10.1145/2699026.2699101,10.1007/978-3-662-43936-4_21,10.1007/s10916-016-0588-0}. In essence, each data record is specified with an access policy based on attributes (e.g., post title and/or specialty), rather than identities, so that only the clients who possess authorized attributes can access the record. For example, a patient may authorize the access of his/her medical record only to senior researchers or doctors specializing in cancer.

Unfortunately, only few studies have explored query authentication for databases where access control is enforced~\cite{10.1007/978-3-540-88313-5_12,10.1007/978-3-642-39256-6_14,10.1145/1066157.1066204} and they suffer from three major limitations. First, they only support simple access control rules at a coarse-grained level. For example,~\cite{10.1145/1066157.1066204} simply does not allow disclosure of data that are outside the query range;~\cite{10.1007/978-3-540-88313-5_12,10.1007/978-3-642-39256-6_14} divide the database space into sub-spaces and enforce access control for each sub-space, based on identities. Second, unlike fine-grained access control, the access control in these existing studies is not cryptographically enforced, which renders the system susceptible to bypasses and SQL injection attacks~\cite{10.1145/2699026.2699101}. Third, the existing authentication techniques distinguish between inaccessible data and non-existent data and, hence, could leak sensitive information beyond what can be deduced from accessible data. For example, suppose a doctor is authorized to access medical records associated with some specific disease only; by knowing the existence of the inaccessible records, he/she may infer the distribution of other diseases in the database.

To overcome these limitations, in this dissertation, we investigate the problem of authenticating relational queries with fine-grained access control. To answer a query, the outsourced database returns those records that not only satisfy the query condition but are also accessible to the client. As with existing query authentication techniques, a cryptographic proof is returned, along with the query results. The key challenge lies in how to protect information confidentiality during query authentication. That is, the proof must be \emph{zero-knowledge}, so that it reveals nothing beyond the accessible records. For example, if no record is returned for a given search key, the client cannot deduce from the proof whether there does not exist a matching record or the matching record is inaccessible to him/her. This is essential in preventing enumeration attacks that exhaustively and iteratively issue queries of overlapping ranges to learn the distribution of search keys in the database. It is also useful to prevent linkage attacks that use such auxiliary information to compromise privacy in associated databases~\cite{10.1145/1749603.1749605}.

As a building block, we first propose a novel access-policy-preserving (APP) signature based on a variant of the \emph{attribute-based signature} (ABS) scheme~\cite{10.1007/978-3-642-19074-2_24}. The main novelty of the APP signature is its dual roles:
\begin{inlineenum}
  \item for authorized clients, it provides a proof of integrity of the data record;
  \item for unauthorized clients, it can be tailored to derive customized signatures to prove the inaccessibility while achieving the zero-knowledge confidentiality.
\end{inlineenum}
We then design the ADS that consists of APP signatures of data records, based on which authenticated query processing algorithms are developed. To address performance issues for range queries and join queries, we further propose AP$^2$G-tree, a grid-index-based ADS that can aggregate APP signatures. Several optimization techniques that are compatible with the original security model or a relaxed one are also developed.

\section{Authenticating {kNN} Queries in Distributed Settings}

To date, most of the authenticated query processing studies are confined to a centralized environment. Due to the large amount of data, the SP can opt to store and process data in a distributed framework. For example, systems such as Spatial Hadoop~\cite{10.1109/icde.2015.7113382} and Geospark~\cite{10.1145/2820783.2820860} provide spatial queries based on a cluster framework. Therefore, distributed spatial query authentication should be developed to satisfy the SP's guarantee of soundness and completeness. However, the challenge is that the ADS should be designed to well fit the distributed setting and facilitate distributed query processing. Moreover, the VO size should be small enough so that the communication overhead between the SP and the client is light and the client's verification time is short.

In this dissertation, we consider the distributed query authentication problem for kNN ($k$ nearest neighbor) queries. We propose a new distributed ADS, i.e., distributed MR-tree. It has local and global layers and perfectly suits the distributed master-slave system. After receiving a kNN query from the client, the master node emits a message to some slaves, which will process the query and generate the partial results and VOs. The reducer then consolidates all the partial results as well as the partial VOs and sends them to the client for result verification. We also propose two optimized algorithms to reduce the VO size. The two optimized algorithms spend more query time in return for smaller VOs. The experiments show that the optimized algorithms outperform the basic algorithm in terms of the VO size and only sacrifice a little query cost. Furthermore, it is demonstrated that the system scales well with the node number in terms of the system throughput.

\section{Dissertation Organization}

The rest of the dissertation is organized as follows. \Cref{chap:related-works} reviews existing studies. \Cref{chap:aggregate-queries} presents the authentication of aggregate queries over set-valued data. \Cref{chap:access-control} studies the authentication of relation queries with fine-grained access control. The authentication of {kNN} queries in distributed settings is investigated in \Cref{chap:knn}. Finally, the dissertation is concluded in \Cref{chap:conclusions}.
