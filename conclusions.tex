\chapter{Conclusions}\label{chap:conclusions}

In this dissertation, we have tackled three query authentication problems, namely authentication of aggregate queries over set-valued data, authentication of relational queries with fine-grained access control, and authentication of {kNN} queries in distributed settings. They enable the DO to outsource the data and query services to an untrusted third party cloud while still maintaining data integrity.
% Furthermore, they address a variety of needs demanded by enterprise customs such as supporting aggregate queries over set-valued data, enforcing fine-grained access control, and using distributed computing paradigms.
Security analysis and performance evaluation show that the proposed solutions and techniques are robust and efficient under a wide range of system settings.

\section{Contributions}

The first contribution of this dissertation is the development of the query authentication algorithms that support aggregate queries over set-valued data. To the best of our knowledge, this is the first work that addresses both integrity and confidentiality for aggregate queries over set-valued data. We propose a privacy-preserving authentication framework and develop a set of privacy-preserving authentication protocols and algorithms for various aggregate queries. Formal security analysis and cost models for the proposed authentication protocols and algorithms are also given. To further enhance the performance of query authentication algorithms, we propose several optimizations. Finally, we perform extensive performance evaluation on real-world datasets. Empirical results demonstrate the feasibility and robustness of our proposals.

The second contribution of this dissertation is the study of combining query authentication and access control. To the best of our knowledge, this is the first work on query authentication for databases with fine-grained access control. We believe that this is a timely study, as enterprise cloud database systems dictate integrity (query authentication) and authorization (access control) at the same time. We propose a novel ABS-based APP signature as the primitive for ADS, together with a grid-index-based tree structure that can aggregate APP signatures for efficient range and join query authentication. To further improve the query performance, we develop several optimization techniques that are either compatible with the original security model or a relaxed one. Furthermore, we conduct a security analysis and empirical study on the authentication performance with respect to various factors such as query range size and database cardinality.

The third contribution of this dissertation is an investigation of authenticated kNN query processing in distributed settings. We propose distributed MR-tree as an ADS to suit the distributed query processing environment. We develop a basic algorithm and two optimization techniques to efficiently process authenticated kNN queries in a distributed fashion. We also conduct extensive experiments to validate the performance of the proposed techniques in terms of query cost, communication overhead, and verification time.

\section{Future Directions}

The studies conducted in this dissertation lie in the intersection of database query processing and system security. Though it can be seen that authenticated query processing plays an important role on providing data integrity assurance to the data outsourcing services, existing solutions suffer two drawbacks. On the one hand, exiting techniques usually incur dramatic even unpractical overhead on the efficiency. On the other hand, only limited query types are supported. Therefore, as the future work, we would like to seek new theoretical refinement to improve their performance, as well as develop authenticated algorithms to support more applications needed by enterprise customs. In the following, we highlight several directions for possible future studies.

\textbf{Accelerating Authenticated Query Processing.}
It can be seen that many query processing procedures and certain cryptographic primitives are highly parallelizable. Thus, we could leverage a various of parallel technologies such as GPGPU to accelerate the computation during query processing at the SP\@. Furthermore, based on what we develop in \Cref{chap:knn}, we plan to design a distributed query authentication system to reduce the query response time and support more query tasks, such as relational queries, spatial queries, and many others. Finally, we could investigate better strategies to balance the time consumption between query processing at the SP and result verification at the clients.

\textbf{Supporting More Complex Query Types.}
Supporting more complex query types is crucial to a broader adoption of the query authentication technologies in the real-life applications.
Based on the proposed privacy-preserving authentication techniques developed in \Cref{chap:aggregate-queries}, we could study the query authentication algorithms to support more complex aggregate queries such as median and percentile. Similarly, we plan to extend the proposed techniques in \Cref{chap:access-control} to support access control enforced analytic queries. General verifiable computation schemes such as zk-SNARKs have been shown to support a wide range of query types but on the expensive of unpractical overheads. In comparison, the ADS-based solutions while efficient only support specific tasks. Therefore, we could adopt the Goldilocks principle by combining these two approaches to support arbitrary query tasks with efficiency. Specially, we plan to investigate the general verifiable computation algorithms with their subtasks accelerated by ADS-based techniques. Finally, we plan to study the query authentication problems for multi-source data and with frequent data updates.

\textbf{Integrity-Assured Searchable Blockchain.}
Blockchain has recently emerged as a promising solution for providing trustworthy storage and computation for decentralized applications. However, in order to ensure the data integrity, one need to join the blockchain network and becomes a full node by maintaining a full replica of the blockchain. Obviously, such a requirement causes heavy burden to many clients especially for embedded and mobile clients. Therefore, we plan to study to offer integrity-assured blockchain query service by utilizing authenticated query processing techniques. Specifically, we intend to study authenticated range and boolean queries over blockchain databases. We also plan to study different query modes, such as time window, sliding window, and subscription queries. On the other hand, blockchain offers a new way to execute trustworthy computation known as smart contract. Thus, we could leverage smart contract to offer authenticated query service directly, handle data updates on behalf of the lightweight DO, or enforce access control.

