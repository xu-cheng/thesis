\chapter{Proof of \texorpdfstring{\cref*{thm:aggregate-queries:sec}}{Theorem~\ref{thm:aggregate-queries:sec}}}%
\label{app:aggregate-queries}

We start by proving the security of the privacy-preserving authentication protocols on multiset operations. Note that the proofs of $sub(\cdot, \cdot)$ and $empty(\cdot)$ are similar to those on set operations in~\cite{10.1007/978-3-642-22792-9_6} and are hence omitted.

\section{Security of \texorpdfstring{$sum(\cdot)$}{sum(\textcdot)} Operation}

\begin{lemma}[Security of $sum(\cdot)$ operation]\label{lem:aggregate-queries:sum}
  Under the bilinear $q$-strong Diffie-Hellman assumption, if the accumulative value returned by $sum(\{X_1,\dots,X_n\})$ passes the client's verification, then the probability of $sum(\{X_1,\dots,X_n\}) \neq acc(\uplus_{i=1}^n X_i)$ is negligible for any PPT adversary.
\end{lemma}

\begin{proof}
  First, we prove that this lemma holds for $n=2$ by contradiction. Support there is a PPT algorithm that computs such a multiset $S = \{y_1, \dots, y_{\ell}\}$, whereas $y_j \notin X_1 \uplus X_2$ for some $1 \le j \le \ell$. This means that
  \begin{align*}
    e(acc(X_1), acc(X_2)) &= e(g^{\prod_{y \in S} (y+s)}, g) \Rightarrow \\
    {e(g, g)}^{\prod_{x \in X_1 \uplus X_2} (x+s) } &= {e(g, g)}^{\prod_{y \in S} (y+s)}
  \end{align*}
  Note that $(y_j + s)  \nmid \prod_{x \in X_1 \uplus X_2} (x+s)$. Therefore, there exists a polynomial $Q(s)$ (computable in polynomial time) of degree $n-1$ and constant $\lambda \neq 0$, such that $\prod_{x \in X_1 \uplus X_2} (x+s) = Q(s)(y_j + s) + \lambda$. Thus, we have
  \begin{align*}
& {e(g, g)}^{(y_j + s)\prod_{1 \le i \neq j \le \ell} (y_i+s)} = {e(g,g)}^{Q(s)(y_j + s) + \lambda} \Rightarrow \\
& {e(g, g)}^{1/(y_j + s)} = {\left[ {e(g,g)}^{\prod_{1 \le i \neq j \le \ell} (y_i+s)} {e(g,g)}^{-Q(s)} \right]}^{\lambda^{-1}}.
  \end{align*}
  Thus, this algorithm can break the bilinear $q$-strong Diffie-Hellman assumption. This proves that our lemma holds for $n=2$.

  Supposing our lemma is true for any $n=k$, where $k\ge 2$, in the following we prove it is also true for $n=k+1$. Let $\varepsilon_k$ denote the event that \textsf{Adv} returns incorrect accumulative value $sum(\uplus_{i=1}^{k} X_i)$ and passes the client's verification. By law of probability, we have:
  \begin{align*}
    \Pr[\varepsilon_{k+1}] &=  \Pr[\varepsilon_{k}]\Pr[\varepsilon_{k+1} | \varepsilon_{k}] +
    \Pr[\varepsilon_{k}']\Pr[\varepsilon_{k+1} | \varepsilon_{k}']\\
                           &\leq \Pr[\varepsilon_{k}] + \Pr[\varepsilon_{k+1}|\varepsilon_{k}'] \\
                           &= \Pr[\varepsilon_{k}] + \Pr[\varepsilon_{2}],
  \end{align*}
  where $\varepsilon'$ denotes the complement of an event. Finally, we have:
  \begin{align*}
    \Pr[\varepsilon_n] \leq (n-1) \Pr[\varepsilon_2]
  \end{align*}
  Since $\Pr[\varepsilon_2]$ is negligible according to our proof for $n=2$, we can conclude that $\Pr[\varepsilon_{n}]$ is negligible as $p\gg n$ in a cyclic multiplicative group $\mathbb{G}$. Hence, this lemma is proved.
\end{proof}

\section{Security of \texorpdfstring{$union(\cdot)$}{union(\textcdot)} Operation}

\begin{lemma}[Security of $union(\cdot)$ operation]\label{lem:aggregate-queries:union}
  Under the bilinear $q$-strong Diffie-Hellman assumption, if the accumulative value returned by $union(\{X_1,\dots,X_n\})$ passes the client's verification, then the probability of $union(\{X_1,\dots,X_n\}) \neq acc(\cup_{i=1}^n X_i)$ is negligible for any PPT adversary.
\end{lemma}
\begin{proof}
  Let $U = \cup_{i=1}^n X_i$. The $union(\cdot)$ protocol consists of three modules
  \begin{inlineenum}
  \item $sub(\widehat{X}_1, U)$, $sub(\widehat{X}_2, U)$, $\dots$, $sub(\widehat{X}_n, U)$;%
    \label{enum:aggregate-queries:union-proof:mod1}
  \item $empty(\{U-\widehat{X}_1, U-\widehat{X}_2, \dots, U -\widehat{X}_n\})$; and%
    \label{enum:aggregate-queries:union-proof:mod2}
  \item SP sends the ECRH hash value ${h_e(U)}^*$ for $U$.%
    \label{enum:aggregate-queries:union-proof:mod3}
  \end{inlineenum}
  Due to the composition property and the security of~\ref{enum:aggregate-queries:union-proof:mod1} and~\ref{enum:aggregate-queries:union-proof:mod2}, we only need to prove the security of~\ref{enum:aggregate-queries:union-proof:mod3}, that is, ${acc(U)}^*$ cannot be forged if ${h_e(U)}^*$ passes the client's verification. Denote $acc^{-1}({acc(U)}^*)$ as the set who has accumulator value ${acc(U)}^*$.

  First, we prove $acc^{-1}({acc(U)}^*)\subseteq U$, i.e., the returned value ${acc(U)}^*=g^{{P(U)}^*}$=$g^{P(U) \cdot Q_U(s)}$, where $Q_U(s)$ is some polynomial. We prove this by contradiction.
  We assume there exists an element $x_j \in \widehat{X}_i$ such that $(x_j + s) \nmid {P(U)}^*$, i.e., ${P(U)}^* = (x_i + s) \cdot Q_i(s) + \lambda_i$, where $Q_i(s)$ is some polynomial and $\lambda_i$ is a constant value. During the execution of protocol $sub(\widehat{X}_i, U)$, by expanding $e(acc(\widehat{X}_i), W_i^*) = e({acc(U)}^*, g)$, the adversary can get:
  \begin{align*}
    e(g^{\prod_{k=1}^{|\widehat{X}_i|}(x_k+s) }, W_i^*)  = {e(g, g)}^{(x_j+s) \cdot Q_i(s) + \lambda_i}.
  \end{align*}
  By dividing $(x_j+s)$ on the exponents of both sides, the adversary can get:
  \begin{align*}
    {e(g, W_i^*)}^{\prod_{k=1}^{j-1}{(x_k+s)} \cdot \prod_{k=j+1}^{|\widehat{X}_i|}{(x_k+s)}} = {e (g, g)}^{Q_i(s) + \lambda_i/(x_j+s)}.
  \end{align*}
  Finally, the adversary can get:
  \begin{align}
&{e(g, g)}^{1/(x_j + s)}  \nonumber \\
    = &{({e(g, W_i^*)}^{\prod_{k=1}^{j-1}(x_k+s) \cdot \prod_{k=j+1}^{|\widehat{X}_i|}{(x_k+s)}} \cdot {e(g, g)}^{-Q_i(s)})}^{1/\lambda_i},\nonumber
  \end{align}
  which violates the bilinear $q$-strong Diffie-Hellman assumption mention in \cref{sec:aggregate-queries:prelim}. Hence, by contradiction, ${P(U)}^*$ must be in the form of $P(U) \cdot Q_U(s)$.

  Next, we prove $U \subseteq acc^{-1}({acc(U)}^*)$, i.e., $Q_U(s)$ can only be a constant value. We prove this by contradiction. For each set $\widehat{X}_i$, ${acc(U - \widehat{X}_i)}^* = g^{P(U-\widehat{X}_i) \cdot Q_U(s)}$. During the execution of $empty(\cdot)$, by expanding $\prod_{i=1}^n e({acc(U-\widehat{X}_i)}^*, F_i^*)=e(g, g)$, the adversary can get:
  \begin{align*}
    \prod_{i=1}^n e(g^{P(U-\widehat{X}_i)}, g^{Q_i(s)}) = e(g, g),
  \end{align*}
  and then
  \begin{align*}
    {e(g,g)}^{Q_U(s)(\prod_{i=1}^n P(U-\widehat{X}_i)\cdot Q_i(s))}=e(g,g).
  \end{align*}
  Finally, the adversary can get:
  \begin{align*}
    {e(g, g)}^{1/Q_U(s)} = {e(g, g)}^{P(U-\widehat{X}_i \cdot Q_i(s))}.
  \end{align*}
  Since $Q_U(s)$ is a polynomial whose order is larger than $1$, the adversary violates the bilinear $q$-strong Diffie-Hellman assumption. Thus, by contradiction, $Q_U(s)$ must be a constant value.

  By the above two arguments, we prove that the accumulative value ${acc(U)}^*$ cannot be forged if it passes the client's verification.
\end{proof}

\section{Security of \texorpdfstring{$times(\cdot, \cdot)$}{times(\textcdot, \textcdot)} Operation}

\begin{lemma}[Security of $times(\cdot,\cdot)$ operation]\label{lem:aggregate-queries:times}
  Under the bilinear $q$-strong Diffie-Hellman assumption, if the accumulative value returned by $times(X,t)$ passes the client's verification, then the probability of $times(X,t) \neq acc(\uplus_{i=1}^t X)$ is negligible for any PPT adversary.
\end{lemma}
\begin{proof}
  The $times(\cdot, \cdot)$ operation relies only on the $sum(\cdot)$ operation, and therefore the proof is similar to that of \cref{lem:aggregate-queries:sum}.
\end{proof}

\section{Security of PA\texorpdfstring{$^2$}{\^{}2} Algorithms}

Following the above lemmas, we now show that the query results returned by the PA$^2$ algorithms cannot be forged.
\aggregatesecuritytheorem*

\begin{proof}
  According to the lemmas on the security of operations $sub(\cdot,\cdot)$, $empty(\cdot)$, $sum(\cdot)$, $union(\cdot)$, $times(\cdot,\cdot)$, the theorem is trivially true due to the composition property.
\end{proof}

