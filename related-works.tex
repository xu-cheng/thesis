\chapter{Literature Review}\label{chap:related-works}

In this chapter, we survey related works on aggregate queries, access control, distributed spatial queries, and authenticated query processing.

\section{Aggregate Queries over Set-Valued Data}

Set-valued data, in which a set of values is associated with an individual, is common in data analytics applications ranging from market basket analysis, to web log mining, to health care research. \citeauthor{10.1145/32204.32219} were the first to extend relational algebra and aggregate functions to set-valued data~\cite{10.1145/32204.32219}. They defined algebraic expressions such as set union, set difference, and aggregation. Since then, data analytics over set-valued data has been extensively studied in various application domains. The most well-known application is association rule mining~\cite{10.1145/170035.170072}. Given a database of sales transactions, each containing a subset of items in the product universe, the objective is to find rules such as ``a user buying item(s) $X$ will probably buy item(s) $Y$''. Although various algorithms are proposed, a fundamental problem in this application is to efficiently compute two aggregate values, namely, the \emph{support} and \emph{confidence} of $X$~\cite{Agrawal:1994:FAM:645920.672836}. The former is the number of transactions that contain $X$, whereas the latter is those in the former that also contain $Y$. With the boom of web search and online advertisement, query log and click stream have become new sources of set-valued data. A variety of aggregate queries have been proposed on these sources for tasks such as website clustering and frequent item identification and counting~\cite{10.14778/2367502.2367508}. Recently, graphs have become another new source of set-valued data. In particular, social networks have contributed various social relations, such as ``friend/unfriend'', ``follow'', ``post/tag'', and user access rights, to set-valued data~\cite{10.1145/1592568.1592585}. Aggregating such data for social network analysis and recommendation has been intensively studied~\cite{10.1145/1367497.1367646}.

\section{Access Control}

Enforcing access control in file systems or database systems has been widely studied in the literature. Traditionally, this is done by employing an \emph{access control list} (ACL), in which a permission manifest is attached to each individual file or data record. However, this method suffers from poor scalability when dealing with massive data or complex access control requirements. To remedy such issues, \emph{role-based access control} (RBAC) is proposed by \citeauthor{10.1109/2.485845}~\cite{10.1109/2.485845}. RBAC implements an access control mechanism over role permissions, user-role, and role-role relationships. This makes it a flexible technology in supporting both \emph{discretionary access control} (DAC) and \emph{mandatory access control} (MAC). However, only static and pre-defined access policies can be supported with RBAC\@. To overcome this limitation and support dynamic, context-aware, and fine-grained access control, \emph{fuzzy identity-based encryption} (Fuzzy IBE)~\cite{10.1007/11426639_27} and, later, \emph{attribute-based encryption} (ABE)~\cite{10.1145/1180405.1180418} are developed. These approaches define an access policy with a complex boolean function over many different attributes to support \emph{attribute-based access control} (ABAC)~\cite{10.1109/ccgrid.2012.92}. There are two categories of ABEs. In \emph{key-policy ABE} (KP-ABE)~\cite{10.1145/1180405.1180418}, each data object is associated with a set of attributes, while users' decryption keys define the access policies with a boolean function over those attributes. In \emph{ciphertext-policy ABE} (CP-ABE)~\cite{10.1109/sp.2007.11}, the access policy is embedded in each data object's ciphertext, while users are given a set of attributes as private keys to present their roles. ABE offers an effective way to enforce fine-grained access control over encrypted data, but it cannot be used to authenticate one's identity. To address this, \emph{attribute-based signature} (ABS)~\cite{10.1007/978-3-642-19074-2_24,10.1145/1755688.1755697} is proposed as a signature scheme to prove one's identity that satisfies certain fine-grained constraints. While ABAC was originally designed for file systems, it has been recently adopted by various cloud databases to support fine-grained access control~\cite{10.1145/2699026.2699101,10.1007/978-3-662-43936-4_21,10.1007/s10916-016-0588-0}.

\section{Distributed Spatial Queries}

There are a large body of research on distributed spatial queries. \citeauthor{10.14778/2536222.2536227} designed the Hadoop-GIS system~\cite{10.14778/2536222.2536227}, which is a spatial data warehousing system and integrates Hive. SpatialHadoop~\cite{10.1109/icde.2015.7113382} adds the traditional spatial indexes to the native Hadoop framework and supports a variety of spatial queries such as range query, kNN, and spatial join. For in-memory computation over cluster machines, GeoSpark~\cite{10.1145/2820783.2820860} has been proposed to support similar spatial queries. More recently, an more efficient in memory spatial analytics system, called Simba~\cite{10.1145/2882903.2915237}, has been proposed. It supports rich spatial queries and has better throughput than that of SpatialHadoop and GeoSpark. Nevertheless, none of the system supports the authenticated query processing to guarantee the integrity of the results.

\section{Authenticated Query Processing}
